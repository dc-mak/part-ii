\documentclass[12pt]{article}

\usepackage{booktabs} % Nice tables
\usepackage{graphicx}
\usepackage{hyperref}
\usepackage{makerobust}
\usepackage{ulem} % Strikethrough
\usepackage[dvipsnames]{xcolor}

\parindent 0pt
\parskip 6pt

\newcommand{\added}[1]{\textcolor{ForestGreen}{#1}}
\newcommand{\done}[1]{\sout{#1}}
\newcommand{\removed}[1]{\textcolor{red}{#1}}

\begin{document}

\begin{center}
\Large
Computer Science Tripos -- Part II -- Progress Report\\[4mm]
\LARGE
Exploring the structure of mathematical theories using graph databases\\[4mm]

\large
Dhruv C.~Makwana, Trinity College

dcm41@cam.ac.uk

\end{center}

\vspace{5mm}

\textbf{Project Supervisor:} Dr.~Timothy G.~Griffin

\textbf{Directors of Studies:} Dr.~Frank Stajano \& Dr.~Sean B.~Holden

\textbf{Project Overseers:} Dr.~David J.~Greaves  \& Prof.~John Daugman

% Main document
\section*{Scheduling}

The project is on schedule. Although the some of the details of the work to be
done (especially regarding the script-to-CSV tool) have changed, most of it
(compiled-to-CSV and library of queries) has been as predicted.

Firstly, work on the compiled-to-CSV tool is further along (with more detailed
models, such as hierarchical modules and type-constructor relations.) than
anticipated, reducing the need for the script-to-CSV tool. The current
implementation suffers from some strange results (thanks to the representation
of modules types/functors) but is overall, working as expected.

Work on the library of queries/tools is further along than expected thanks to
pre-existing solutions (APOC: Awesome Procedures on Cypher) to some of the
graph algorithm and query ideas considered. What remains to be done is either
implementing a new metric or an existing one, aiming for better performance.

Lastly, work on the script-to-CSV tool has been temporarily suspended. This is
because the model, as it stands, of Coq's proof-objects is sufficiently detailed
to run some queries on and develop a library for. The only missing features are
(a) inclusion of notation in the model and (b) appropriate de-duplications of
module types (and dealing appropriately with opaque \texttt{:=} and
transparent \texttt{<:} ascriptions).

\section*{Unexpected Difficulties}

Understanding the Coq AST and API has been even more challenging than
anticipated.  The project is large and poorly documented, so a lot of time was
spent trying to figure out how to find and chain appropriate AST
transformations.

Once Coq was better understood, it became apparent that the full Coq AST was
impractical to work with for this project and a easier alternative would be to
process \texttt{glob} files, (global reference) which will provide the
aformentioned missing features.

\section*{Accomplished Work}

Accomplished work is detailed in Table \ref{table:milestones}.

\sout{Strikethroughs} represent work done. \textcolor{ForestGreen}{Green}
represents additions to the original schedule presented in the proposal and
\textcolor{red}{red} shows work removed or rescheduled.

Exploring queries and starting the library was delayed (to January) as it
became clear the compiled-to-CSV tool could provide more information than
originally expected (and thus more time was reallocated to that).

CoqPL is a workshop (to which this project was submitted). It was removed from
the schedule when the proposal submission of this project was rejected.  The
time for this was allocated to getting ahead on writing the final dissertation.

\begin{table}[p]
\centering
\caption{Milestones: Updated (\done{Done}, \removed{Removed}, \added{Added})}
\label{table:milestones}
\begin{tabular}{ll}
    \toprule

    Date & Milestone \\

    \midrule

	% October
    21-10-2016	&	\done{Complete Project Proposal} \\[6pt]
          
	% November
    04-11-2016 & \done{Finish a prototype compiled-to-CSV tool.} \\ 
               & \done{Get familiar with Neo4j Cypher.} \\
               & \done{Understand how to use the Coq parser.} \\[6pt]

    18-11-2016 &	\done{Refine compiled-to-CSV tool: tests and documentation.} \\
               &	\removed{Explore queries possible and start the library.} \\
               &	\done{Begin work on translating Coq constructs from proof-scripts.} \\[6pt]

	% December
    02-12-2016	&	\removed{Finish a prototype script-to-CSV tool.} \\
                & \added{\sout{Refactor compiled-to-CSV tool, add new features.}} \\[6pt]

    16-12-2016	&	\removed{Test and document script-to-CSV tool.} \\
               & \added{\sout{Test and document refactored compiled-to-CSV tool.}} \\[6pt]

    30-12-2016	&	\done{Begin work on integrating tools into one workflow.} \\[6pt]

	% January
	13-01-2017	&	\done{Stabilise and document whole project so far.} \\
              &	\removed{Prepare presentation for CoqPL Conference.} \\
              & \added{\sout{Write Introduction, Preparation chapters.}} \\[6pt]

    27-01-2017	&	\removed{Look at SSReflect and evaluate changes to be made.} \\
               &	\added{\sout{Explore tools and queries possible, start library.}} \\[6pt]

	% February
    10-02-2017	&	\removed{Incorporate changes from feedback/new features.} \\
               &	\added{Implement new graph query/algorithm.} \\[6pt]

    24-02-2017	&	Test and document the new features. \\[6pt]

	% March
    10-03-2017	&	\removed{Write Introduction, Preparation and Implementation chapters.} \\
                & \added{Write Implementation chapters.} \\
               &	\added{Implement script-to-CSV tool.} \\[6pt]

    24-03-2017	&	Fix bugs/unexpected problems. \\[6pt]
    
	% April
    07-04-2017	&	Write Evaluation and Conclusion chapters. \\[6pt]

    21-04-2017	&	Fix bugs/unexpected problems. \\[6pt]
    
	% May
	05-05-2017	&	Complete Dissertation (references, bibliography, appendix, formatting). \\[6pt]

  \bottomrule

\end{tabular}
\end{table}

\end{document}
