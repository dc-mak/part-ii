\chapter{Full Model}\label{chapter:fullmodel}

Here is the full translation of the Coq AST to kinds and subkinds.

\begin{minted}[linenos,obeytabs=true,tabsize=2]{ocaml}
let kind_of_constref =
let open Decl_kinds in function
| IsDefinition def -> ("definition", Some (match def with
  | Definition -> "definition"
  | Coercion -> "coercion"
  | SubClass -> "subclass"
  | CanonicalStructure -> "canonical_structure"
  | Example -> "example"
  | Fixpoint -> "fixpoint"
  | CoFixpoint -> "cofixpoint"
  | Scheme -> "scheme"
  | StructureComponent -> "projection"
  | IdentityCoercion -> "coercion"
  | Instance -> "instance"
  | Method -> "method"))
| IsAssumption a ->
  ("assumption", Some (match a with
  | Definitional -> "definitional"
  | Logical -> "logical"
  | Conjectural -> "conjectural"))
| IsProof th ->
  ("proof", Some (match th with
  | Theorem -> "theorem"
  | Lemma -> "lemma"
  | Fact -> "fact"
  | Remark -> "remark"
  | Property -> "property"
  | Proposition -> "proposition"
  | Corollary -> "corollary"))

let kind_of_ind ind =
let (mib,oib) = Inductive.lookup_mind_specif (Global.env ()) ind in
if mib.Declarations.mind_record <> None then
  let open Decl_kinds in
  begin match mib.Declarations.mind_finite with
  | Finite -> "recursive_inductive"
  | BiFinite -> "recursive"
  | CoFinite -> "corecursive"
  end
else
  let open Decl_kinds in
  begin match mib.Declarations.mind_finite with
  | Finite -> "inductive"
  | BiFinite -> "variant"
  | CoFinite -> "coinductive"
  end

let get_constr_type typ =
  Names.KerName.to_string (Names.MutInd.user typ)

let kind_of_gref gref = 
if Typeclasses.is_class gref then
  ("class", None)
else
  match gref with
  | Globnames.ConstRef cst ->
  kind_of_constref (Decls.constant_kind cst)

  | Globnames.ConstructRef ((typ, _), _) -> 
  ("type_constructor", None)

  | Globnames.IndRef ind -> 
  ("inductive_type", Some (kind_of_ind ind))

  | Globnames.VarRef _ ->
  assert false


let kind_of_obj = function
| G.Node.Gref gref -> 
  kind_of_gref gref
| G.Node.Module modpath ->
      ("module", match modpath with
      | Names.ModPath.MPbound _ -> Some "bound"
      | Names.ModPath.MPdot _ -> Some "module"
      | Names.ModPath.MPfile _ -> Some "file")
\end{minted}

\chapter{Timings}\label{chapter:timings}

Below is the table of measurements I conducted of different tools that aim to
help a user understand a Coq library on a Surface Pro 3 (Intel Haswell i7-4650U
1.7-3GHz, 8GB RAM, 512GB SSD) running Fedora 24 inside VirtualBox 5.1.16 on
Windows 10 (Creator's Update). I rebooted the VM between measurements to reduce
the effects of caching.

\begin{table}[h]

  \centering
  \small
  \begin{tabular*}{\textwidth}{@{\extracolsep{\fill}} lrrrrrrr}

    \toprule

    \textbf{Timing} & \textbf{Test 1}	& \textbf{Test 2}	& \textbf{Test 3}	&
    \textbf{Test 4}	& \textbf{Test 5}	& \textbf{Mean} & \textbf{Std. Dev.} \\

    \midrule

    DOC: U & 7886    & 7885    & 7879    & 7875    & 7879    & 7880.8    & 4.6 \\
    DOC: S & 307     & 303     & 288     & 295     & 300     & 298.6     & 7.4 \\
    DOC: R & 10018   & 9997    & 9997    & 10017   & 10007   & 10007.2   & 10.3 \\

    DEP: U & 1363    & 1365    & 1357    & 1359    & 1363    & 1361.4    & 3.3 \\
    DEP: S & 2307    & 2299    & 2314    & 2292    & 2297    & 2301.8    & 8.7 \\
    DEP: R & 4263    & 4254    & 4257    & 4257    & 4256    & 4257.4    & 3.4 \\

    DPD: U & 6869    & 6868    & 6872    & 6878    & 6876    & 6872.6    & 4.3 \\
    DPD: S & 654     & 631     & 639     & 646     & 650     & 644       & 9.1 \\
    DPD: R & 9396    & 9381    & 9390    & 9404    & 9382    & 9390.6    & 9.7 \\

    CRT: U & 79544   & 79529   & 79510   & 79526   & 79517   & 79525.2   & 12.9 \\
    CRT: S & 2522    & 2518    & 2528    & 2522    & 2523    & 2522.6    & 3.6 \\
    CRT: R & 98146   & 98173   & 98165   & 98156   & 98159   & 98159.8   & 10.1 \\

    2DT: U & 1272235 & 1272244 & 1272264 & 1272250 & 1272239 & 1272246.4 & 11.3 \\
    2DT: S & 2517    & 2537    & 2519    & 2517    & 2515    & 2521      & 9.1 \\
    2DT: R & 1588302 & 1588305 & 1588304 & 1588305 & 1588299 & 1588303   & 2.5 \\

    ANL: U & 255915  & 255909  & 255922  & 255914  & 255927  & 255917.4  & 7.1 \\
    ANL: S & 49568   & 49565   & 49567   & 49565   & 49572   & 49567.4   & 2.9 \\
    ANL: R & 371877  & 371884  & 371876  & 371890  & 371886  & 371882.6  & 6.0 \\

    \bottomrule

  \end{tabular*}
  \normalsize

  \bigskip 

  \caption{Table of execution time measurements of tools that aim to help a
    user understand a Coq library. All timings are in milliseconds. DOC=coqdoc,
    DEP=coqdep, DPD=dpd2graph, CRT=creation, 2DT=dpd2dot, ANL=analysis, U=user,
    S=system, R=real. Mean and sample standard deviation rounded to the nearest
    decimal point.}

\end{table}

\chapter{Sample Code}\label{chapter:sample}

Here is the dependency collection code (recursing down the AST).

\begin{minted}[linenos,obeytabs=true,tabsize=2]{ocaml}
let collect_dependance gref =
  match gref with
  | Globnames.VarRef _ -> assert false

  | Globnames.ConstRef cst ->
      let cb = Environ.lookup_constant cst (Global.env()) in
      let cl = match Global.body_of_constant_body cb with
         Some e -> [e]
	| None -> [] in
      let cl = match cb.Declarations.const_type with
        | Declarations.RegularArity t -> t::cl
        | Declarations.TemplateArity _ ->  cl in
      List.fold_right collect_long_names cl Data.empty

  | Globnames.IndRef i ->
      let _, indbody = Global.lookup_inductive i in
      let ca = indbody.Declarations.mind_user_lc in
      Array.fold_right collect_long_names ca Data.empty

 | Globnames.ConstructRef (i,_) -> 
      let _, indbody = Global.lookup_inductive i in
      let ca = indbody.Declarations.mind_user_lc in
      (* So a constructor and its type are linked, BUT WRONG WAY AROUND *)
      add_inductive i (Array.fold_right collect_long_names ca Data.empty)
\end{minted}

Here is a small excerpt of how edges are formatted and corrected.

\begin{minted}[linenos,obeytabs=true,tabsize=2]{ocaml}
  let out_edge fmt _g e =

    (* incorporate src & dst types, flip if constructor & ind & match
     * TO FIX WRONG WAY DEPENDENCY LINK FROM SEARCHDEPEND.ML4 *)
    let src, dst, rel_type =
      let matches typ n =
        let dirname, name = G.Node.split_name (G.Node.obj n) in
        get_constr_type typ = dirname ^ "." ^ name in

      let src, dst = G.Edge.src e, G.Edge.dst e in

      match G.Node.obj src, G.Node.obj dst with
      | G.Node.Gref (Globnames.ConstructRef ((typ, _), _)),
        G.Node.Gref (Globnames.IndRef _) when matches typ dst ->
        dst, src, "CONSTRUCTED_BY"
      | G.Node.Module _, G.Node.Module _ ->
        dst, src, "CONTAINS"
      | G.Node.Module _, G.Node.Gref _ ->
        src, dst, "CONTAINS"
      | _, _ ->
        src, dst, "USES" in

    let edge_attribs =
      [ ("type", rel_type) ; ("weight", string_of_int (G.Edge.nb_use e))] in

    (* NOTE: Flipped src and dst - src USED_BY dst <==> dst USES src*)
    Format.fprintf fmt "E: %d %d [%a];@."
        (G.Node.id dst)
        (G.Node.id src)
        pp_attribs edge_attribs
\end{minted}

Here is the analysis of the CoqRegExp library.

\begin{minted}[linenos,obeytabs=true,tabsize=2]{R}
suppressMessages(library(RNeo4j))
suppressMessages(library(igraph))
library(tictoc)

# Connect to DB instance
cat("Connecting to database... "); tic()
graph <- startGraph(
  "http://localhost:7474/db/data/",
  username="neo4j",
  password="Neo4j")
time <- toc(quiet=TRUE); time <- time$toc - time$tic
cat(sprintf("done. (%.2fs)\n", time))

# Grab nodes...
cat("Getting nodes (definitions and proofs) "); tic()
nodes <- cypher(graph, "
  MATCH (obj:definition),
				(m:module)-[:CONTAINS]->(obj)
  RETURN obj.objectId AS id,
         obj.name AS label,
         obj.path AS title,
         \"triangle\" AS shape,
				 m.objectId AS parent
  UNION
  MATCH (obj:proof),
				(m:module)-[:CONTAINS]->(obj)
  RETURN obj.objectId AS id,
         obj.name AS label,
         obj.path AS title,
         \"square\" AS shape,
				 m.objectId AS parent")

# and edges.
cat("and edges... ")
edges <- cypher(graph, "
  MATCH (src)-[edge]->(dst)
  WHERE (src:definition OR src:proof) AND (dst:definition OR dst:proof)
  RETURN src.objectId AS from, dst.objectId AS to, edge.weight AS weight")
time <- toc(quiet=TRUE); time <- time$toc - time$tic
cat(sprintf("done. (%.2fs)\n", time))

# Constructing graph
cat("Constructing igraph... "); tic()
d_ig <- graph_from_data_frame(edges, directed=TRUE, nodes)
ig <- graph_from_data_frame(edges, directed=FALSE, nodes)
time <- toc(quiet=TRUE); time <- time$toc - time$tic
cat(sprintf("done. (%.2fs)\n", time))

# Calculate metrics on nodes
node_metric <- function(algorithm, graph) {
  cat(sprintf("%s ... ", algorithm)); tic()

  result <- switch(algorithm,
                   PageRank = page_rank(graph)$vector,
                   Betweenness = betweenness(graph),
                   Closeness = closeness(graph))

  time <- toc(quiet=TRUE); time <- time$toc - time$tic
  cat(sprintf("done. (%.2fs)\n", time))

  return(result)
}

# Compute PageRank (proofs and definitions)
nodes$pagerank <- node_metric("PageRank", d_ig)

# Compute Betweenness (proofs and definitions)
nodes$betweenness <- node_metric("Betweenness", d_ig)

# Compute Closeness (proofs and definitions)
nodes$closeness <- node_metric("Closeness", d_ig)

# Assign a cluster to each node
assign_cluster <- function(algorithm, nodes, graph) {
  cat(sprintf("%s clustering... ", algorithm)); tic()

  label <- switch(algorithm,
                  "Fast Modularity" = "modularity",
                  Betweenness = "betweenness",
                  "Label Propagation" = "label_prop")

  communities <- switch(algorithm,
                        "Fast Modularity" = cluster_fast_greedy(graph),
                        Betweenness = cluster_edge_betweenness(graph),
                        "Label Propagation" = cluster_label_prop(graph))

  memb <- data.frame(id = communities$name)
  memb[[label]] <- communities$membership
  nodes <- merge(nodes, memb)

  time <- toc(quiet=TRUE); time <- time$toc - time$tic
  cat(sprintf("done. (%.2fs)\n", time))

  return(nodes)
}

# Cluster modularity (proofs and defintions) 
# Needs UNDIRECTED graph
nodes <- assign_cluster("Fast Modularity", nodes, ig)

# Only run on small graphs.
nodes <- assign_cluster("Betweenness", nodes, d_ig)

# Cluster label_prop (proofs and defintions)
nodes <- assign_cluster("Label Propagation", nodes, d_ig)

# Put back into database
cat(sprintf("Setting properties:\n\t%s\n\t%s\n\t%s\n\t%s\n\t%s\n\t%s\n",
            "definition_proof_pagerank",
            "definition_proof_betweenness",
            "definition_proof_closeness",
            "definition_proof_modularity",
            "definition_proof_edge_betweenness",
            "definition_proof_label_prop")); tic()
set = "
  MATCH (obj { objectId : toInt({OBJID}) })
  SET obj.definition_proof_pagerank = toFloat({PGR}), 
            obj.definition_proof_betweenness = toFloat({BTW}),
            obj.definition_proof_closeness = toFloat({CLOSE}),
            obj.definition_proof_modularity = toInt({MODGROUP}),
            obj.definition_proof_edge_betweenness = toInt({EBTW}),
            obj.definition_proof_label_prop = toInt({LBL_PRP})
"
transaction <- newTransaction(graph)
progressBar <- txtProgressBar(min=0,
                              max=nrow(nodes),
                              char='=',
                              width=80, 
                              style=3)
for (i in 1:nrow(nodes)) {
  row <- nodes[i,]
  appendCypher(transaction,
               set,
               OBJID=row$id,
               PGR=row$pagerank,
               BTW=row$betweenness,
               CLOSE=row$closeness,
               MODGROUP=row$modularity,
               EBTW=row$edge_betweenness,
               LBL_PRP=row$label_prop)
  setTxtProgressBar(progressBar, i)
}
close(progressBar)
cat("Commiting transaction... ")
commit(transaction)
time <- toc(quiet=TRUE); time <- time$toc - time$tic
cat(sprintf("done. (%.2fs)\n", time))
\end{minted}

Here is the visualisation of the CoqRegExp library.

\begin{minted}[linenos,obeytabs=true,tabsize=2]{R}
suppressMessages(library(RNeo4j))
suppressMessages(library(igraph))
suppressMessages(library(visNetwork))
library(tictoc)

# Connect to DB instance
cat("Connecting to database... "); tic()
graph <- startGraph(
  "http://localhost:7474/db/data/",
  username="neo4j",
  password="Neo4j")
time <- toc(quiet=TRUE); time <- time$toc - time$tic
cat(sprintf("done. (%.2fs)\n", time))

# Grab nodes...
cat("Getting nodes (definitions and proofs) "); tic()
nodes <- cypher(graph, "
  MATCH (obj:definition),
				(m:module)-[:CONTAINS]->(obj)
  RETURN obj.objectId AS id,
         obj.path + '.' + obj.name AS title,
         \"triangle\" AS shape,
         obj.definition_proof_pagerank AS pagerank,
         obj.definition_proof_betweenness AS betweenness,
         obj.definition_proof_closeness AS closeness,
         obj.definition_proof_edge_betweenness AS edge_betweenness,
         obj.definition_proof_label_prop AS label_prop,
         obj.definition_proof_modularity AS modularity
  UNION
  MATCH (obj:proof),
				(m:module)-[:CONTAINS]->(obj)
  RETURN obj.objectId AS id,
         obj.path + '.' + obj.name AS title,
         \"square\" AS shape,
         obj.definition_proof_pagerank AS pagerank,
         obj.definition_proof_betweenness AS betweenness,
         obj.definition_proof_closeness AS closeness,
         obj.definition_proof_edge_betweenness AS edge_betweenness,
         obj.definition_proof_label_prop AS label_prop,
         obj.definition_proof_modularity AS modularity")

# and edges.
cat("and edges... "); tic()
edges <- cypher(graph, "
  MATCH (src)-[edge]->(dst)
  WHERE (src:definition OR src:proof) AND (dst:definition OR dst:proof)
  RETURN src.objectId AS from, dst.objectId AS to, edge.weight AS weight")
time <- toc(quiet=TRUE); time <- time$toc - time$tic
cat(sprintf("done. (%.2fs)\n", time))

# Output visualisations
visualise <- function(nodes, edges, filename, layout_opts,
											edge_opts, skipIgraph=FALSE) {
  cat(sprintf("Outputting %s... ", filename)); tic()

  g <- visNetwork(nodes, edges, width="1600px", height="1600px") %>%
       visInteraction(navigationButtons=TRUE,
											dragNodes=FALSE, zoomView=FALSE)

	g <- do.call(visEdges, append(list(g), edge_opts))

  if (!skipIgraph) {
    g <- do.call(visIgraphLayout, append(list(g), layout_opts))
  } else {
    visLayout(g, randomSeed=layout_opts$randomSeed)
  }

  visSave(g, file = filename)

  time <- toc(quiet=TRUE); time <- time$toc - time$tic
  cat(sprintf("done. (%.2fs)\n", time))
  return(g)
}

# Bucket
bucket <- function(x) { return(cut(log(0.001+x), 10, labels=FALSE)) }

# Non DrL layouts
other_layout <- list(randomSeed=1492)

# Edge options
edge_opts <- list(arrows="middle", color=list(opacity=1), dashes=FALSE)

# Modularity (FR)
nodes$value <- bucket(nodes$betweenness)
nodes$group <- nodes$modularity
other_layout$layout <- "layout_with_fr"
edge_opts <- list(color=list(opacity=1), dashes=FALSE)
visualise(nodes, edges, "direct.html", other_layout, edge_opts)

# Grid graph
nodes$value <- cut(nodes$betweenness, 10, labels=FALSE)
nodes$group <- nodes$modularity
other_layout$layout <- "layout_on_grid"
edge_opts <- list(arrows="middle", color=list(opacity=0.7), dashes=TRUE)
visualise(nodes, edges, "grid.html", other_layout, edge_opts)

# Hierarchical graph
nodes$value <- bucket(nodes$betweenness)
nodes$group <- nodes$modularity
other_layout$layout <- "layout_with_sugiyama"
edge_opts <- list(color=list(opacity=1), dashes=TRUE)
visualise(nodes, edges, "hierarchical.html", other_layout, edge_opts)


# Construct a hierarchical network
cat("Getting nodes (modules) "); tic()
modules <- cypher(graph, "
  MATCH (obj:module)
  RETURN obj.objectId AS id,
         coalesce(obj.path + '.', '') + obj.name AS title,
         \"circle\" AS shape,
         NULL AS pagerank,
         NULL AS betweenness,
         NULL AS edge_betweenness,
         NULL AS closeness,
         NULL AS modularity,
         NULL AS label_prop,
         9 AS value")

# and edges.
cat("and edges... ")
contains <- cypher(graph, "
  MATCH (src)-[edge:CONTAINS]->(dst)
  WHERE (src:module) AND (dst:module OR dst:definition OR dst:proof)
  RETURN src.objectId AS from, dst.objectId AS to,
				 10000 * edge.weight AS weight")
time <- toc(quiet=TRUE); time <- time$toc - time$tic
cat(sprintf("done. (%.2fs)\n", time))

# Modularity (FR, modules)
nodes$value <- bucket(nodes$betweenness)
nodes$group <- nodes$modularity
modules$group <- max(nodes$group)+1
other_layout$randomSeed=440
edge_opts <- list(color=list(opacity=1), dashes=FALSE)
visualise(rbind(nodes, modules),
          contains,
          "module.html",
          other_layout,
          edge_opts,
          skipIgraph=TRUE)

\end{minted}
