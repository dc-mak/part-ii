% vim: set spell spelllang=en_gb tw=80:
\chapter{Conclusions}

\prechapter{%
  I have just described a tool that allows users to understand a Coq library more
  effectively. It does so by translating Coq libraries to Neo4j graph databases.
  I gave examples of new questions that can now be answered by querying the
  library and showed how network-analysis algorithms and visualisations can
  provide new insights into the structure of a Coq library.  I used a variety of
  technologies, (Coq, Neo4j, OCaml and R) tools (APOC, RNeo4j, igraph,
  visNetwork) and algorithms (centrality, clustering and other statistics) in
  this project. Its feature set, performance and output on real-world Coq
  libraries show that this project has potential to be useful to the Coq
  community.
}%

\section{In Hindsight}

I learnt \emph{a lot} during this project.  Working with existing systems
\emph{and} creating something novel at the same time was difficult to juggle
mentally. I found that I became better at learning how to use different
frameworks and managing my time effectively as the project progressed.
Initially, I was intrigued and fascinated by the project idea, suggested by the
project supervisor. After I had completed preliminary preparations, the novelty,
difficulty and usefulness of the concept had taken root.

However, Coq's lack of good, structured documentation made implementing the
project frustrating.  It was easy to get stuck in the details and spend hours
staring at the entire Coq compiler code base just to figure out how to do
something.  Sometimes, having several options for how to proceed but no clear
way of comparing between them made implementation difficult. It was rarely
\emph{writing} the code which was the issue, but \emph{knowing what to aim for}.
It was in those moments that having an experienced, focused and clear supervisor
was the most useful.

I found project management and organisation easy and immensely helpful in
providing a useful structure for guiding implementation. A good grasp of Unix,
Git and programming languages proved to be essential to executing this project
as smoothly as possible. Evaluating the project gave me the chance to switch
from \emph{developing} to \emph{using} and explore what was possible with the
project in its final form. The only thing I would do differently with the
benefit of hindsight would be to make sure my development workflow (Makefiles,
Linux VM, Git, editors, continuous-integration builds) was set up from the start
so it would have been easier to do so and useful for longer.

\section{Future Work}

Immediate extensions to this project could come from fleshing out the model
further, to include tactics, notation and other aspects of the Coq system.  More
ambitious extensions could involve somehow presenting some information to a user
\emph{as they are working} on something, or using data science techniques to
analyse the output model as a means to providing new insights or a more helpful
compiler which can suggest how to (re)structure a project.  Although very useful
and easily interpretable in the context of mathematical theories, this project's
core concept could also be applied to any programming language and its
libraries, to provide new ways of becoming familiar with and understanding
increasingly complex software projects.
