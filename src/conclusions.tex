% vim: set spell spelllang=en_gb tw=80:
\chapter{Conclusions}

%\prechapter{%
%  Born from the need to detangle mathematical theories, this project uses graph
%  databases and a host of related approaches to help bring order to the concepts
%  being explained. It attempts and succeeds at going further than tools such as
%  coqdoc, coqdep, dpdgraph and others. A variety of technologies, (Coq, Neo4j,
%  Java, OCaml and R) tools (APOC, RNeo4j, igraph, visNetwork) and algorithms
%  (centrality, clustering and other statistics) make up this project. Its
%  feature set, performance and output on real-world packages show that this
%  project has a genuine potential to be useful in the Coq community.
%}%

\section{In Hindsight}

\emph{A lot} was learned throughout the course of this project. Working with
existing systems \emph{and} creating something novel can be difficult to juggle
mentally: the cognitive load of imagining what is possible within the given
frameworks and timescales is a skill, one which was refined over the course of
the project. Initially, the idea, suggested by the project supervisor, seemed
fascinating and intriguiging. After the preliminary preparations, the novelty,
difficulty and usefulness of the concept had taken root.

However, at several instances throughout implementation, it was easy to get
stuck in the details and spend hours staring at the entire Coq compiler code
base just to figure out how to get the name of the module an object is contained
in or how to convert some part of the AST to a string for use elsewhere; the
lack of good, structure documentation made the whole endeavour even more
frustrating at times. Other times it was simply the bewildering choice of how
best to proceed next given many options and no clear way of comparing between
them. It was rarely \emph{writing} the code which was the issue, but
\emph{knowing what to aim for}. It was in those moments that having an
experienced, focused and clear supervisor who reiterated the overarching vision
of the project was the most useful.

Project management and organisation tended to be easy and helped immensely in
providing a useful structure for guiding implementation. A good grasp of Unix,
Git and programming languages proved to be essential to executing this project
as smoothly as possible. Evaluating the project gave a chance to switch from
\emph{developing} to \emph{using} and explore what was possible with the project
in its final form, and how it could be improved and extended given more time.

\section{Future Work}

Immediate extensions to this project could come from fleshing out the model
further, to include tactics, notation and other aspects of the Coq system.
More ambitious extensions could involve somehow presenting some information
to a user \emph{as they are working} on something, or using data science
techniques to analyse the output model as a means to providing new insights or
a more helpful compiler which can suggest how to (re)structure a project.
Although very useful and easily interpretable in the context of mathematical
theories, this project's core concept could also be applied to any programming
language and its libraries, to provide new ways of becoming familiar with and
understanding increasingly complex software projects.
