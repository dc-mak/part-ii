\chapter{Introduction}

{\sf \color{red}
  The Introduction should explain the \textbf{principal motivation} for the
  project.  \textbf{Show how the work fits into} the broad area of surrounding
  Computer Science and \textbf{give a brief survey} of previous related work.
  It should \textbf{generally be unnecessary to quote at length} from technical
  papers or textbooks. If a simple bibliographic reference is insufficient,
  consign any lengthy quotation to an appendix.
}

This dissertation offers a solution to problems regarding the presentation of
mathematics. Firstly, these problems and the specific systems involved are
described. Then, the aims of the project are stated; in later chapters, details
of the prepration and implementation carried out are expounded. Lastly, the
chapters on evaluation and conclusion provide evidence for the success of the
project, reflections on the process and suggestions for future work.

\section{Problem}

Mathematics textbooks aimed at professionals/researchers follow a well-established
rhythm: define some constructions and some properties on them and prove
theorems on both, with lemmas, corollaries and notation interspersed
throughout. Such a presentation is concise but limiting: it is linear; it forces
the reader to keep track of dependencies such as implicit assumptions, previously
defined results and the types and conventions behind any notation used;
and it offers little opportunity to consider and compare different approaches
for arriving at a result (i.e.\ number of assumptions, number of steps, some
notion of the importance of a result such as number of uses by later
results).

With the increasing popularity of interactive proof-assistants such as
Coq~\cite{Coq:manual} and Isabelle~\cite{nipkow2002isabelle}, many mathematical
theories (such as the formidably large Feit-Thompson Odd Order
Theorem~\cite{peterfalvi2000oot, bender1994oot}) have
been~\cite{gonthier2013oot} or are being translated and formalised into
machine-checked proof-scripts. However, these proof-scripts on their own
inherit the same disadvantages as the aforementioned textbooks, as well as some
new ones: they are usually more verbose and explicit and are primarily designed
for automation/computation than readability. The former (usually out of
necessity to convey to the computer the intended meaning) leads to unnecessary
``noise'' in the proof and the latter departs from the vocabulary or flow of a
natural-language presentation.

The database world is currently experiencing a tremendous explosion of creativity
with the emergence of new data models and new ways of representing and querying
large data sets. \emph{Graph databases} have been developed to deal with highly
connected data sets and path-oriented queries. That is, graph databases are
optimised for computing transitive-closure and related queries, which pose a
huge challenge for traditional, relational databases.

\section{Solution}

A graph-based approach to the representation and exploration of the structure of
proof-objects would  be a far more natural expression of the complex
relationships (i.e.\ chains of dependencies) involved in
constructing mathematical theories. Questions such as ``What depends on this
lemma and how many such things are there?'' or ``What are the components of
this definition?'' could thus be expressed concisely (questions which are not
even expressible with standard relational databases systems such as
SQL). A popular graph database, Neo4j~\cite{neo4j} with an expressive query
language \emph{Cypher} will be used for this project.

\section{Coq Proof-Assistant}

The Coq proof-assistant -- implemented in OCaml -- can be viewed as both 
a system of logic -- in which case it is a realisation of the \emph{Calculus of
Inductive Constructions} -- and as a \emph{dependently-typed} programming
language. Its power and development are therefore most-suited and often geared
towards \emph{large scale} developments.

On the logical side, Coq lays claim to projects such as the Four-Colour
Theorem~\cite{gonthier2008formal} (60,000 lines) and the aformentioned
\emph{Feit-Thompson} theorem (approximately $170,000$ lines, $15,000$
definitions and $4,200$ theorems) are feats of modern software-engineering.

On the programming language side, Coq has served as the basis for many equally
fantastic projects. The \emph{CompCert Verified C
Compiler}~\cite{leroy2012compcert} demonstrates the practical applications of
theorem-proving and dependently-typed programming by implementing and proving
correct an optimising compiler for the C programming language.
\emph{DeepSpec}~\cite{pierce2016science}, a recently announced meta-project,
aims to integrate several large projects such as \emph{CertiKOS} (operating
system kernels, \emph{Kami} (hardware), \emph{Vellvm} (verifying LLVM) and many
more in the hopes to provide complete, \emph{end-to-end} verification of
real-world systems.

\section{Neo4j Database and the Cypher Language}

Neo4j is a graph database system implemented in Java. Traditional, relational
database theory and systems are designed with the goal of storing and
manipulating information in the form of \emph{tables}. As such, working with
highly interconnected data, such as social network graphs is best tackled with
the alternative approach of \emph{graph databases}.

Briefly, a (directed) \emph{graph} is defined as $G = (V, E)$ where $V$ is a set
of vertices or \emph{nodes} and $E \subseteq V \times V$ is a set of edges or
\emph{relationships} between two nodes. A \emph{graph database} is an OLTP
(online transaction processing, meaning operated upon live, as data is
processed) database management system with CRUD (create, read, update and delete)
operations acting on a graph data model. Relationships are therefore promoted to
first-class citizens and can be manipulated and analysed.

\subsection*{Cypher: An Illustrated Example}

\emph{Cypher} features heavy use of pattern-matching in an ASCII-art inspired
syntax.  The following (slightly contrived but hopefully illuminating) example
in Listing \ref{lst:cypherexample} illustrates some of the key strengths of
graph-based modelling using Cypher. 

Suppose we have a very popular puppy named ``Cliff'' looking for the nearest
and most familiar children (for this example, a person under the age of six) to
play with.

To see how Cliff (indirectly) likes/knows this child, we bind \emph{path} to the
result of the \emph{shortestPath} query.  For the path itself, we start with a
node following this structure: \mintinline{cypher}{(var:label {attrib: val})}.
We then have a \emph{labeled, transitive} relationship (explicitly limiting our
search to paths of up to length four) expressed as an arrow with a label
\texttt{-[..]->}. As such, we can discard any paths with relationships we do not
want (e.g.\ \texttt{HATES}). 

To filter based on more complex logic (than possible by pattern-matching directly
on labels and attributes) we can express the requirement that the age of a dog by
the name of Cliff be less than or equal to two (and similarly for the age of the
child) in the \mintinline{cypher}{WHERE} clause.

Lastly, we return the path and order the results by proximity as a row of
results, renaming the column of the child's name to simply ``name''.

\begin{listing}[tb]

  \label{lst:cypherexample}
  \caption{Example Cypher Query}

  \begin{minted}[frame=lines]{cypher}
MATCH path = shortestPath(
  (puppy:dog {name: "Cliff")-[:LIKES|:KNOWS*..4]->(child:person))
WHERE puppy.age <= 2 AND child.age < 6
RETURN path,
       child.name AS name,
       ORDER BY other.distance_from_clifford
  \end{minted}

\end{listing}

\section{Related Work}

Some existing tools offer part of the solutions.

Coq-dpdgraph
(\href{http://github.com/Karmaki/coq-dpdgraph}{\texttt{github.com/Karmaki/coq-dpdgraph}})
is a tool which analyses dependencies between \emph{compiled} Coq proofs. As
such, desirable information about notation, tactics, definitions and the
relationship between a type and its constructors is lost.

Coqdep is a utility included with Coq which analyses dependencies \emph{at the
module level} by tracking {\tt Require} and {\tt Import} statements.

Coq SerAPI
(\href{http://github.com/ejgallego/coq-serapis}{\texttt{github.com/ejgallego/coq-serapis}})
is a work-in-progress library and communication protocol for Coq designed to
make low-level interaction with Coq easier, especially for IDEs. It has a
starting point for gathering some statistics of proof-objects in a project.

\section{Aims of the Project}
\label{intro:aims}

This project aimed to:

\begin{itemize}
\item represent Coq libraries as Neo4j graph databases, which involved
  \begin{itemize}
  \item exploring and choosing the correct model
  \item converting and extending existing code to output CSVs
  \item writing new programs to extract extra information \\
      (omitted from other, existing tools)
  \item writing new programs to automate database creation; and to
  \end{itemize}

\item create a library of Neo4j queries, intended
  \begin{itemize}
  \item to highlight the structure and relationship between proof-objects
  \item by coalescing and implementing several graph-related metrics.
  \end{itemize}
\end{itemize}

\section{Summary}

An explanation of the problems which conventional presentations of mathematics
suffer from was given, with \emph{graph databases} proposed as a solution. 
Existing tools were mentioned and the requirements for a successful project
were listed.
